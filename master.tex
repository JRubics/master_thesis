% vim: ft=tex expandtab

\documentclass[12pt]{report}
\usepackage[utf8]{inputenc}
\usepackage[serbian]{babel}
\usepackage[backend=biber, style=numeric, sorting=none]{biblatex}
\usepackage{csquotes}
\usepackage[a4paper, top=20mm, bottom=20mm, left=30mm, right=15mm, headheight=10mm, headsep=5mm, footskip=9mm]{geometry}
\usepackage{fancyhdr}
\usepackage{float}
\usepackage{fontspec}
\usepackage[acronym]{glossaries}
\usepackage{graphicx}
\usepackage{hyperref}
\usepackage{titlesec}
\usepackage{ragged2e}
\usepackage{setspace}

\hypersetup{hidelinks}

\setmainfont{Times New Roman}

\tolerance=1
\emergencystretch=\maxdimen
\hyphenpenalty=10000
\hbadness=10000

\addbibresource{literature.bib}

\pagestyle{fancy}

\renewcommand{\chaptermark}[1]{\markboth{#1}{}}
\renewcommand{\sectionmark}[1]{\markright{\arabic{section}.\ #1}}
\renewcommand{\baselinestretch}{1.4}
\newcommand{\normal}{
\tolerance=1
\emergencystretch=\maxdimen
}

\makeatletter
\newcommand\frontmatter{
    \cleardoublepage{}
    \pagenumbering{Roman}
    \setlength{\parskip}{0pt}
}

\setsansfont{Arial}

\newcommand\mainmatter{
    \cleardoublepage{}
    \pagenumbering{arabic}
    \setlength{\parskip}{2mm}
    \titleformat{\chapter}{\normalfont\Large\bf\sffamily\raggedleft}{\thechapter.}{12pt}{}
}
\makeatother

\titlespacing*{\chapter}{0mm}{58mm}{10mm}
\titleformat{\chapter}{\normalfont\Large\bf\sffamily}{\thechapter.}{12pt}{}
\titleformat{\section}{\normalfont\large\bf\sffamily}{\thesection}{12pt}{}
\titleformat{\subsection}{\normalfont\bf\sffamily}{\thesubsection}{12pt}{}
\titleformat{\subsubsection}{\normalfont\bf\sffamily}{\thesubsubsection}{12pt}{}
\setcounter{secnumdepth}{4}

\fancypagestyle{plain}{}
\fancyhead[L]{}
\fancyhead[R]{\leftmark}
\fancyfoot{}
\fancyfoot[R]{\thepage}

\raggedbottom{}

\begin{document}

\frontmatter{}

\renewcommand{\MakeUppercase}[1]{#1}
\tableofcontents

\listoffigures

\chapter*{Skraćenice}
\chaptermark{Skraćenice}
\begin{tabular}{ l l }
    \textbf{API} & -- \textit{Application programming interface}, Interfejs za programiranje aplikacija \\
\end{tabular}

\mainmatter{}
\chapter{Uvod}
U današnje vreme\cite{docker}

\begin{figure}[H]
    \includegraphics[width=\linewidth]{images/vm.png}
    \caption{Razlika između arhitekture Virtuelne Mašine a arhitekture kontejnera}
\end{figure}

\chapter{Teorijske osnove}
U ovom poglavlju će biti ukratko objašnjeni svi koncepti i alati koji su bili potrebni za izradu projekta o kojem ovaj rad govori. NABROJ
\section{Open source razvoj}
\section{Github}
\section{Github akcije - mozda spojiti sa prethodnim}
\section{Python}

Python \cite{python} je programski jezik opšte namene, spada u grupu programskih jezika visokog nivoa i interpretira se. Dizajniran je tako da bude izrazito čitljiv, što se postiže velikom upotrebom belina (eng.\ \textit{whitespace character}). Pogodan je za razvoj različitih projekata pošto podržava više programskih paradigmi:

\begin{enumerate}
    \item Proceduralnu
    \item Objektno orijentisanu
    \item Funkcionalnu
    \item Strukturalnu paradigmu
\end{enumerate}

Takođe, prilagođen je za korišćenje u radličitim okruženjima na različitim operativnim sistemima.

Python se prvi put spominje krajem 1980-tih kao naslednik ABC programskih jezika. Dizajnirao ga je Guido van Rossum i objavio prvu verziju 1991. godine. Python 2.0, koji je izdat 2000. godine je znatno unapređenje prve verzije, koje se do skoro koristilo.  Poslednja velika revizija urađena je 2008. godine kada je nastao Python 3.0. Zbog toga što Python 3 nije u potpunosti kompatibilan sa prethodnim verzijama, podrška za Python prestala je u januru 2020. godine.

Umesto da poseduje svu svoju funkcionalnost u jezgru OS-a. Python je dizajniran tako da  bude veoma proširiv, što ga je učinilo veoma pogodnim za dodavanje programabilnih interfejsa postojećim aplikacijama.

\section{Python paketi, PyPI}

\section{Poetry}
Poetry \cite{poetry} je alat otvorenog koda (eng.\ \textit{open-source}) za kreiranje Python paketa i upravljanje paketima koje Python projekat koristi(eng.\ \textit{dependency management}). Pomaže korisnicima tako što umesto njih instalira i menja verzije  bibliotekama od kojih projekat zavisi, a koje korisnik mora pretkodno da specificira. Poetry je takođe znatno olakšao kreiranje, pakovanje i objavljivanje paketa na PyPI - repozitorijum softvera za Python programski jezik i time omogućio lakše deljenje Python projekata sa drugim korisnicima tog jezika.

Prva verzija alata objavljena je 28.02.2018. godine. Verzije programskog jezika Python koje su podržane su 2.7 i 3.4+. Ideja je da Poetry radi podjednako dobro na različitim platformama, između ostalog na Linux-u, Windows-u i OSX-u.

\subsection{Podešavanje Poetry-ja}
Za razliku od drugih Python alata za upravljanje paketima od kojih projekat zavisi, umesto pip-a - instalera za Python pakete, Poetry koristi svoj specifičan način za instalaciju. Instalator doda Poetry u korisnički direktorijum, tako da može da se koristi sa bilo kojom verzijom Python-a. Omogućena je i instalacija Poetry-ja korišćenjem pip-a, ali se ne proporučuje iz dva razloga:

\begin{enumerate}
    \item može dovesti do konflikta sa drugim sistemskim fajlovima
    \item otežava održavanje konzistentnosti pre korišćenju različitih verzija Python-a i različitih virtuelnih okruženja (eng.\ \textit{virtual environments})
\end{enumerate}

\subsection{Kreiranje Python projekta koji koristi Poetry}
Nakon isntalacije alata, moguće je kreirati projekat koji koristi Poetry za upravljanje paketima ili dodati Poetry u postojeći projekat i time kreirati virtuelno okruženje u kojem će se izvršavati naredne akcije. U oba slučaja će se kreirati \textit{pyproject.toml} datoteka koja čuva sve bitne informacije o projektu. U početku ova datoteka sadrži samo osnovne informacije o projektu:

\begin{samepage}
    \begin{verbatim}
    [tool.poetry]
    name = "poetry-demo"
    version = "0.1.0"
    description = ""
    authors = ["Name Surname <email>"]

    [tool.poetry.dependencies]
    python = "*"

    [tool.poetry.dev-dependencies]
    pytest = "^3.4"
    \end{verbatim}
\end{samepage}

Informacije o samom projektu, njegovom autoru, verziji, licenci, opisu, dokumentaciji, itd. nalaze se u prvom segmentu - \texttt{[tool.poetry]}. Informacije o paketima koje projekat koristi nalaze se u naredne dve sekcije - \texttt{[tool.poetry.dependencies]} i \texttt{[tool.poetry.dev-dependencies]}, gde se u prvoj od ove dve sekcije nalaze informacije o paketima koji se koriste u produkcionom okruženju, a u drogoj informacije o paketima koji se koriste u toku razvoja projekta. Pri instalaciji novog paketa, njegov naziv i verzija će se dopisati u jednu od ove dve sekcije, ili u obe ako je to potrebno.

Ovu datoteku je moguće ručno menjati, ali se sa njom najčešće interaguje pozivanjem poetry komandi u konzoli. Neke od tih komandi su:

\begin{itemize}
    \item \texttt{poetry add} - dodaje novi paket u \textit{pyproject.toml} datoteku
    \item \texttt{poetry remove} - briše paket iz projekta
\end{itemize}

Komande koje takođe interaguju sa \textit{pyproject.toml} datotekom, ali je ne menjaju su:

\begin{itemize}
    \item \texttt{poetry install} - instalira pakete definisane u \textit{pyproject.toml} datoteci
    \item \texttt{poetry update} - ažurira pakete u skladu sa verzijama koje pišu u \textit{pyproject.toml} datoteci
    \item \texttt{poetry lock} - zaključava pakete u prijektu
    \item \texttt{poetry check} - proverava ispravnost \textit{pyproject.toml} datoteke
\end{itemize}

Za pokretanje komandi u virtuelnom okruženju opsisanom u \textit{pyproject.toml} datoteci koristi se komanda:

\begin{itemize}
    \item \texttt{poetry run}
\end{itemize}

Za sve ostale detalje najbolje je pogledati dokumentaciju pomoću komande:

\begin{itemize}
    \item \texttt{poetry help}
\end{itemize}

\subsection{Kreiranje, pakovanje i objavljivanje Python paketa}
Da bi se Python projekat mogao objaviti, potrebno je kreirati arhivu sa svim potrebnim konfiguracijama za njega. To se postiže upotrebom komande:

\begin{itemize}
    \item \texttt{poetry build}
\end{itemize}

Nakon toga, paket je spreman za objavljivanje. U projektu se pojavi novi direktorijum sa nazivom \texttt{dist} i u njemu se nalaze dve datoteke:

\begin{enumerate}
    \item \texttt{poetry-demo-0.1.0-py2.py3-none-any.whl}
    \item \texttt{poetry-demo-0.1.0.tar.gz}
\end{enumerate}

Repozitorijum na kojem se paket objavljuje se podešava pomoću komande:

\begin{itemize}
    \item \texttt{poetry config}
\end{itemize}

Paket se objavljuje pozivanjem komande:

\begin{itemize}
    \item \texttt{poetry publish}
\end{itemize}

U tom momentu, paket postaje dostupan na izabranom repozitorijumu i svako ko ima pristup repozitorijumu može da instalira paket. Najčešće se paketi objavljuju na PyPI repozitorijumu.

\section{Bash}

\chapter{Specifikacija i implementacija projekta}
\section{pjisp-assignment-template}
\section{pjisp-template-name}
\subsection{način imenovanja repo-a}
\section{pjisp-diff}
\section{poetry-publish action}
Koristi se za publishovanje pjisp-diff i pjisp-template-name
\section{Error code u smoke-test-u - zasto je potreban?}


\chapter{Primeri korišćenja}
\chapter{Diskusija i zakljuci}
\section{Da li su github akcije spremne za produkciju?}
\chapter{Literatura}
\sloppy
\printbibliography[heading=none]

\end{document}
